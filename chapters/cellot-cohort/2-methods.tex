% !TeX root = ../../main.tex
\section{The 4i drug response platform}
In total, we profiled 47 patients between the years of 2018 and 2022.
Patients arrive to the hospital, a biopsy is taken and then processed with the 4iDRP pipeline (Figure \ref{fig:cellot-cohort-overview}a).
For each cell, we used 4i to measure 65 morphological and intensities of tagged proteins that mark for essential cancer processes (Figure \ref{fig:cellot-cohort-overview}b), resulting in a high-dimensional, multiplexed dataset of over 190M cells (Figure \ref{fig:cellot-cohort-overview}c).
We developed a cohort normalization approach by leveraging cell line references, which we measured with each patient sample (Supplementary Figure \ref{fig:cellot-cohort-normalization}).
Our dataset contains two replicate samples originating from the same patient, which were analyzed independently.
We find that the two replicate samples share the highest similarity in their single-cell data both in control and drug-treatment conditions (Supplementary Figure \ref{fig:cellot-cohort-normalization}c).
We observed that our cohort of 37 melanoma samples was composed of individuals with highly personalized clinical profiles, including differing biopsy sites, melanoma subtypes, experienced treatment lines, and the genetic alteration driving the disease (Figure \ref{fig:cellot-cohort-overview}d). 
Traditionally, deriving biological findings or clinical hypotheses based on ex-vivo drug screening from such a small and diverse cohort would have been exquisitely challenging for solid tumors \cite{williams2022}.
We hypothesized that the patient-to-patient heterogeneity and single-cell heterogeneity within individual patient sample had been one of the primary challenges of the field to derive consistent biological findings from clinical samples of advanced cancer patients.

\begin{figure}[h!]
  \centering
  \includegraphics[width=\textwidth]{figures/cellot-cohort/overview.pdf}
  \caption{
    Predicting single-cell responses of progressed melanoma patients.
    From sample collection to machine learning to evaluation.
    a) 4i DRP pipeline for a single sample from tumor profiler.
    A tumor biopsy is taken from a cancer patient, prepared,
    and exposed, in parallel, to 56 unique treatments, including control (1-3).
    Cell morphology and marker intensities are imaged (4).
    An in-house imaging processing pipeline performs cell segmentation
    extracts marker intensity levels and morphology features (5,6).
    Extracted features are selected, standardized, normalized, and batch corrected (7).
    Modeling and validation (8) of single-cell responses in two settings: learning individual responses and predicting responses of \emph{incoming} samples
    b) 4i-profiled markers are selected from known key signaling pathways in cancer.
    c) In total, 47 patients are profiled, 65 cellular features are extracted, 56 total drug responses are measured, each containing about 3000 cells. In total over 190M cells are profiled.
    d) The melanoma cohort represents a diverse, heterogenous set of samples
    across biopsy location, cancer subtype, number of lines of treatment, and (known) driver mutation.
  }\label{fig:cellot-cohort-overview}
\end{figure}

\subsection{Data}

The extracted features are then further normalized and processed to remove batch effects and standardize the range of values. First, wells with an abnormally low number of cells ($< 50$) are considered low-quality and are removed.
The range of feature values is then standardized with a quantile rescaling computed on the individual morphological features and the joint intensity features.
After, the level of background light intensity for each plate is computed and removed.
A secondary control measuring background light intensity, i.e., without a probe, is included in each plate. 
In order to standardize the intensity level of the protein markers across plates, the intensity features of the secondary control are computed and cells on the plate with intensity features below the 75th quantile are clipped.
An additional quality control is then applied to cells that are inflated for no marker intensity.
Cells are removed in which four or more intensity features are measured as zero.
Finally, a variance stabilizing monotonic log + 1 transform is performed to mitigate the influence of any extreme outliers in both the morphological and intensity features.

Due to challenges stemming from the real-time processing of these cancer patients, including time constraints, variations in biopsy size and availability, available materials, etc, a subset of samples, features, and treatments is taken to construct a standardized dataset for the purposes of the machine learning tasks.
In all that follows, a subset of the full dataset is taken consisting of 47 samples, 33 features and 35 treatments.



\subsection{Training of IID tasks}
Heterogeneous single-cell perturbation responses are computed for each treatment using \textsc{CellOT} \cite{bunne2023}.
As we are unable to observe the state of a cell both before and after treatment, we rely on \textsc{CellOT} to recover the individual cellular responses that transformed the set of observed untreated (control) cells into the set of observed treated cells.
As discussed in Chapter \ref{ch:cellot}, \textsc{CellOT} utilizes neural optimal transport theory to learn a neural network parameterization of this optimal transport map, allowing us to make predictions on cells that are not available at training time.

We demonstrate the ability of \textsc{CellOT} to learn the responses tumor cells take to each treatment, individually for each patient. The cells of each sample are split into a 75/10/15 train/valid/test split, computed individually for each treatment. We train the model using cells from the train split, perform validation and any intermediate analyses on the valid split, and only communicate results using predictions made on the test split. Since all splits are sampled from the same sample in an independent and identically distributed (IID) manner, we refer to this setting as the IID setting.

We use the default architecture and hyperparameter choices of \textsc{CellOT}, networks have a width of 64 and depth of 4, convexity on the transporting potential is enforced with a regularization term with equal weight to the loss, we use the Adam optimizer \cite{kingma2014} with a learning rate of 1e-4, a beta of (0.5, 0.9), an inner loop of 10 steps and perform the outer loop for 100000 steps, a number picked as a conservative estimate to ensure convergence. Since these networks are relatively small, they can be comfortably trained on CPUs and are trained in less than 2 hours. 

\subsection{Training of OOD tasks}
In this experiment, we demonstrate how we can learn to generalize a set of observed treatment responses to incoming, unseen samples. To do so,  for each sample, for each treatment, we train a \textsc{CellOT} model on the composite set of all other samples and then predict the responses of the held-out sample. This is often referred to as a hold-one-out experiment and is also an example of an out-of-distribution (OOD) prediction task since the distribution of cellular states in the training set (the cohort) is different from the test set (the holdout sample). We train all models as we had done in the IID experiments, using the same training splits and evaluating using the same set of test cells to keep the experiments comparable. We furthermore introduce an additional evaluation metric by computing the distance of the \textsc{CellOT} predictions made in the IID and OOD setting.

\subsection{Conditional CellOT}
When training models to handle the prediction of responses across a cohort of samples, as is done in the OOD setting, meta information not necessarily contained in the cell feature space may be informative towards the nature of cellular responses. For instance, some targeted therapies are designed to treat patients with a specific driver mutation, and thus, we can expect that patients with these driver mutations may respond differently than those without. While the base model of \textsc{CellOT} is only able to make predictions from cellular features, \cite{bunne2022} described an approach to condition predictions on meta information.

We train a conditional \textsc{CellOT} model using one-hot encodings of both the individual’s known driver mutation and the number of lines of treatment applied. The former is included to handle the differential responses of targeted therapies, while the latter is included since lines of treatment indicate a degree of the acquired immunity of tumor cells.

\subsection{Comparisons}
We compare the performance of \textsc{CellOT} to several baselines, including two auto-encoder baselines. scGen \cite{lotfollahi2019} and the conditional auto-encoder (cAE) \cite{lopez2018} and a simple baseline that predicts responses as the average between the two conditions. While \textsc{CellOT} applies treatment effects directly on the observed cell features themselves, scGen and cAE both rely on auto-encoders to learn compressed representations of cells and apply treatment effects on these representations. For each, we use a two-layer encoder and decoders with a width of 32 and a bottleneck width of 8. We train with the same data splits and use the default Adam parameters for a total of 100000 iterations.
