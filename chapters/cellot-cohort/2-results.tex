\section{Results}

\subsection{IID performance}

\subsection{Associations}

\subsection{OOD performance}

\subsection{Future progression status of incoming samples}
In this experiment we demonstrate the clinical utility of predicting single-cell perturbation responses of incoming patients. We aim to understand if we can predict whether a patient will progress within 90 days using predicted single-cell responses, i.e. without access to treated cell states. To ensure that we select treatments that are relevant to the situation of individual patients, we limit ourselves to the treatments that had been prescribed to samples by the medical board associated with the Tumor Profiler project. From the total cohort, 22 samples had been prescribed small molecule treatments. Of these 22, we removed three samples due to quality control: mykokig, had evidence of unfiltered artifacts in its observed treated cells [SUPPLEMENTAL], motamuh, as it represents an extreme outlier in IID and OOD response prediction tasks [SUPPLEMENTAL] , and mecygyr, as it had showed evidence of immediate progression at the start of the study ( < 5 days). We simulate the clinic environment by splitting the patients into a train (75\%) and test (25\%) set, wherein we have access to the control and treated cells of all patients in the training set, and only have access to control cells of patients in the testset.

Our prediction model works as follows: first, single-cell responses are clustered and patients are represented as frequencies over these clusters, and then an SVM classification model is trained to predict the binary progression status using the patient representations. To construct the patient representations of the training set, a KMeans clustering is computed using the single-cell treatment responses from all patients in the training set, as predicted using the IID CellOT models (SECTION XXX). We perform a soft assignment of each cell to the k cluster centers, via a softmax transformation over the negative distances to each cluster center, to smooth cluster assignments and help include a notion of cluster similarities. After this step, each cell is now represented as a probability over the set of cluster centers. A patient-level representation is constructed by averaging the cluster assignments over all cells in each patient. For test set patients, since we assume we do not have access to treated cell states, we use the OOD CellOT predictions instead of the IID predictions. 

Since this is a balanced classification task, we report accuracy as our metric and consider 50\% a trivial lower bound. For each model setting we perform 100 75\%-25\% train-test splits, i.e. in each split, 4 of the 19 samples are heldout. We include a model that predicts the progression stats at 90 days with meta information at the sample level, specifically a random forest over the subtype, lines of treatment, and known driver mutation, represented as the concatenation over one-hot encodings of each category. We furthermore compare our approach to a similar approach that uses predicted treated states, control cells, and single-cell responses predicted by the baselines (SUPPLEMENT).
