\section{Introduction}
Inferring the molecular responses of cells to perturbations is a core challenge in single cell biology.
A key application lies in personalized cancer treatments wherein the cells that compose a tumor are heterogenous in both structure and response.
Understanding the response of these tumor cells would help select effective treatments, identify relevant biomarkers, or understand the drug resistance or evasion.
While many technologies exist to measure individual cell states, they typically require that the profiled cell is destroyed.
This complicates the learning of such heterogeneous perturbation responses as we are unable to operate in a traditional machine learning paradigm, in which we have access to a set of paired input-output observations, i.e.
the same cell profiled in both the control and treated states.
Instead, we only have access to unpaired distributions of treated and untreated cells.

In this work, we demonstrate how our recently-introduced framework, CellOT, uses optimal transport to infer treatment outcomes of a heterogeneous observational clinical cohort study.
CellOT accomplishes this by learning a neural network-based parameterization of the optimal transport map that couples the observed control and treated distributions.
Concretely, we apply CellOT to learn and predict the multiplexed responses of biopsied cells from 43 metastatic melanoma patients profiled with iterative indirect immunofluorescence imaging (4i).
Using patient-specific maps we reveal otherwise hidden patterns of signaling pathway modulation associated with driver mutations and metastasis sites upon combination therapy treatment with BRAFi \& MEKi.
Finally, we apply CellOT to predict cellular responses of targeted therapies from unseen patients and outcompete currently available methods.

% TODO: major reword this
Cancer remains a significant global health challenge, responsible for substantial morbidity and mortality despite advances in treatment and early detection \cite{}.
The key complication of cancer and its treatment lies in the heterogeneous nature of its tumors.
Tumors are an abnormal mass of cells that have, by way of genetic, epigenetic, and phenotypic variations, evaded the natural biological defenses that control cell growth.
This unchecked growth is what causes dangerous, often fatal, complications in the subject.
Since the mechanisms that cause cancer can arise from a myriad of sources, tumors tend to be specific to each individual.
Specifics of the tumor microenviornment, underlying genetics, patient history, etc,
result in varied therapy responses and complicate the development of treatment strategies effective on some general population.
Precision medicine, a developing approach that tailors treatment to the individual characteristics of each patient's disease,
is emerging as a promising approach to address the individual nature of diseases like cancer.
The proper implementation of precision medicine, however, requires not only sophisticated technologies capable of profiling the relevant molecular and cellular observations,
but also computational methodology to model these complex interactions.

To address these challenges, the Tumor Profiler project, a public-private partnership between the University of Zürich, ETHZ, the University Hospitals Zürich and Basel, and Roche, initiated three observational clinical studies focusing on different oncological indications.
These studies aimed to assess the feasibility of integrating state-of-the-art omics platforms into the clinical routine for cancer care, thus enhancing the personalization of cancer treatment \cite{}.
A key innovation developed through this initiative is the 4i Drug Response Profiling Platform (4iDRP).
The 4iDRP platform represents a cutting-edge approach to measuring the modulation of critical cancer signaling pathways at the single-cell level in response to ex-vivo treatments with FDA-approved cancer drugs.
By utilizing multiplexed imaging technology, the platform provides clinicians with detailed, personalized drug response profiles, facilitating more informed treatment decisions \cite{}.
The 4iDRP platform integrates several advanced technologies, including liquid handling robotics, automated high-content fluorescence microscopy, and high-performance cluster computing.
This combination allows for the efficient processing, measurement, and analysis of various patient sample types, such as core-needle biopsies and resections, within a rapid two-week turnaround time \cite{}.
This capability is crucial for the timely adaptation of treatment strategies based on individual patient responses.

% TODO: update to how cellot does this
In this chapter we demonstrate the feasibility and clinical utility of the 4iDRP platform by profiling drug responses in melanoma patients.
Melanoma, known for its aggressive nature and resistance to treatment, provides a challenging yet ideal context for validating the platform's effectiveness \cite{}.
By measuring the acute modulation of signaling pathways, such as the MEK and MTOR pathways, in single cells derived from patient biopsies, the 4iDRP platform offers a high-resolution view of drug effects at the cellular level.
Patient biopsies provide a unique and valuable resource for reverse translational research, which aims to leverage clinical findings to inform and guide basic research.
By using actual patient samples, researchers can gain insights into the real-world effectiveness of treatments and the biological mechanisms underlying drug responses.
This approach allows for the identification of biomarkers and therapeutic targets directly relevant to patient outcomes \cite{}.
Moreover, patient biopsies can reveal the heterogeneity within tumors, offering a more accurate representation of the disease compared to traditional cell line models.
This real-world data can then be used to refine preclinical models and develop more effective, personalized treatment strategies, ultimately accelerating the translation of laboratory discoveries into clinical practice \cite{}.
Furthermore, this study highlights the potential of using clinical samples for reverse translational research.
Reverse translational research involves leveraging clinical findings to inform and refine basic research, thereby creating a feedback loop that accelerates the discovery and development of new therapeutic strategies \cite{}.
The detailed drug response profiles generated by the 4iDRP platform can provide valuable insights into the mechanisms of drug action and resistance, informing future research and clinical practice.
The study involved the collection and processing of specimens from enrolled patients, which were then dissociated into single-cell suspensions and assessed for viability and tumor cell content.
High-quality samples were subjected to ex-vivo drug treatments, followed by multiplexed imaging to measure the expression of key protein markers involved in cancer signaling pathways \cite{}.
The data generated were analyzed using sophisticated computational tools, including the in-house computer vision platform TissueMaps and the deep learning algorithm CellOT.
These tools enabled the accurate prediction of single-cell drug responses, providing a comprehensive dataset for further analysis \cite{}.
Treating progressed melanoma patients presents significant challenges due to the aggressive and resilient nature of the disease.
Melanoma is notorious for its ability to metastasize rapidly and develop resistance to conventional therapies.
This resistance is often driven by genetic mutations, such as those in the BRAF and NRAS genes, which promote tumor growth and survival even in the presence of targeted treatments \cite{}.
Additionally, the heterogeneity within melanoma tumors complicates treatment, as different subclones within a tumor may respond differently to the same therapy \cite{}.
This intra-tumor heterogeneity necessitates comprehensive profiling and personalized treatment plans to effectively target all cancerous cells.
Furthermore, patients with advanced melanoma often have complex clinical histories and have undergone multiple lines of treatment, which can further complicate the management of their disease \cite{}.
These challenges highlight the need for innovative approaches, such as the 4iDRP platform, to better understand and overcome the mechanisms of drug resistance in melanoma.
Through this research, we aimed to establish a robust framework for integrating the 4iDRP platform into routine clinical practice, thereby enhancing the personalization of cancer treatment.
By combining high-resolution single-cell data with clinical parameters and multi-omics descriptions, the 4iDRP platform holds the potential to transform cancer care, offering more precise and effective treatment options for patients \cite{}.
This introduction sets the stage for the detailed results presented in the following sections, which underscore the platform's capability to provide valuable insights into patient-specific drug responses and its potential to guide personalized therapy in oncology.


