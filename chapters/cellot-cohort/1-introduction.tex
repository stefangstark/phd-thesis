\section{Introduction}
Cancer remains a significant global health challenge, responsible for substantial morbidity and mortality despite advances in treatment and early detection \cite{}.
The key complication of cancer and its treatment lies in the heterogeneous nature of its tumors.
Tumors are an abnormal mass of cells that have, by way of genetic, epigenetic, and phenotypic variations, evaded the natural biological defenses that control cell growth.
This unchecked growth is what causes dangerous, often fatal, complications in the subject.
Since the mechanisms that cause cancer can arise from a myriad of sources, tumors tend to be specific to each individual.
Specifics of the tumor microenviornment, underlying genetics, patient history, etc,
result in varied therapy responses and complicate the development of treatment strategies effective on some general population.
Precision medicine, a developing approach that tailors treatment to the individual characteristics of each patient's disease,
is emerging as a promising approach to address the individual nature of diseases like cancer.
The proper implementation of precision medicine, however, requires not only sophisticated technologies capable of profiling the relevant molecular and cellular observations,
but also computational methodology to model these complex interactions.

To address these challenges, the Tumor Profiler project, a public-private partnership between the University of Zürich, ETHZ, the University Hospitals Zürich and Basel, and Roche, initiated three observational clinical studies focusing on different oncological indications.
These studies aimed to assess the feasibility of integrating state-of-the-art omics platforms into the clinical routine for cancer care, thus enhancing the personalization of cancer treatment \cite{}.
A key innovation developed through this initiative is the 4i Drug Response Profiling Platform (4iDRP).
The 4iDRP platform represents a cutting-edge approach to measuring the modulation of critical cancer signaling pathways in response to ex-vivo treatments with FDA-approved cancer drugs,
profiling protein marker abundancies and morphological features of treated states at single-cell resolution.
Through the integration of several advanced technologies, including liquid handling robotics, automated high-content fluorescence microscopy, and high-performance cluster computing,
the platform exhibits an efficient processing, measurement, and analysis of samples within a rapid two-week window \cite{},
a time-frame crucial for the development of treatment strategies based on individual patient responses.

In this chapter
we aim to leverage clinical findings to inform and guide basic research, a strategy known as reverse translational research.
Patient biopsies provide a unique and valuable resource for such reverse translational research,
where insights into the real-world efficacy and underlying biological mechanisms can go beyond traditional but limiting cell line models.
These real-world findings can then be used to refine preclinical models torwards the development of more effective, personalized treatment strategies, ultimately accelerating the translation of laboratory discoveries into clinical practice \cite{}.

We focus here on a cohort of melanoma patients.
Melanoma is notorious for its ability to metastasize rapidly and develop resistance to conventional therapies.
The treatment of progressed melanoma patients is challenging due to the aggressive and resilient nature of the disease.
This resistance is often driven by genetic mutations, such as those in the BRAF and NRAS genes, which promote tumor growth and survival even in the presence of targeted treatments \cite{}.
Furthermore, patients with advanced melanoma often have complex clinical histories and have undergone multiple lines of treatment that can further complicate the management of their disease \cite{}.
Another key complication torwards the treatment of melanoma derives from its heterogeneity.
Tumors are composed of \emph{subclones} of cancer cells, which may have differential responses to the same therapy \cite{}.
This intra-tumor heterogeneity necessitates comprehensive profiling and personalized treatment plans to effectively target cancerous cells.

We address these challenges by combining
the powerful \textsc{CellOT} prediction framework with the 4iDRP platform
to better understand and overcome the mechanisms of drug resistance in melanoma.
The 4iDRP offers a high-resolution view of drug effects at single-cell resolution, by measurement of the acute modulation of signaling pathways, such as the MEK and MTOR pathways, in cells derived from patient biopsies.
Then, we use \textsc{CellOT} to learn and predict the individual cellular responses to each therapy of interest.
We show that learned resposnes are able to capture response heterogeneity and are more descriptive than currently available approaches.
We go on to demonstrate the ability of the approach to predict treatment responses of incoming samples, i.e. without access to their response profiles,
demonstrate how relevant clinical information can be incorporated into these predictions,
and show that predicted resposnes can help identify at-risk patients.
% TODO: rephrase this a bit more
By combining high-resolution single-cell data with clinical parameters and multi-omics descriptions, the 4iDRP platform holds the potential to transform cancer care, offering more precise and effective treatment options for patients \cite{}.
We hope to set the stage for the detailed results presented in the following sections, which underscore the platform's capability to provide valuable insights into patient-specific drug responses and its potential to guide personalized therapy in oncology.
