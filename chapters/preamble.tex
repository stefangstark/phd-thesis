% !TeX root = ../main.tex

\pdfbookmark[1]{Abstract}{Abstract}
\begingroup
\let\clearpage\relax
\let\cleardoublepage\relax
\let\cleardoublepage\relax

\chapter*{Abstract}
Cellular states are complex -- their underlying genomic sequence, RNA transcription levels, accessibility of chromatin regions, protein levels, chemical activity, etc, work in tandem to determine the behavior of a cell and its responses to changes in its environment. 
Initially, the study of cellular populations relied on bulk profiling technologies
%which average individual cellular signals across population into a single observation, and
that made it difficult to study heterogeneous populations.
However, this last decade has seen, and continues to see, an explosion in the development of single-cell profiling technologies that allow individual cellular states to be observed at scale.
While these technologies revolutionized the study of cellular biology, they are still typically limited in the sense that they consume or otherwise destroy profiled cells.
This means that an individual cell can only be measured once.
As a result, these technologies are only able to measure projections of the underlying \textit{holistic} cellular state into a single observable domain, measuring, for instance, either a cell's RNA transcript abundance or the protein levels.
Furthermore, modeling temporal processes, such as perturbation responses, remain challenging, as these complex behaviors must be recovered without access to some set of paired "input-output" states.

This thesis is broadly concerned with the development and application of machine-learning based methods to reconcile this single-observation limitation.
The typical modern single-cell experiment will generate two (or more) datasets, which we can consider sampled from distributions over cellular states, $\mu$ and $\nu$, that share some non-trivial relationship.
We aim to recover this relationship by learning to \emph{align} these cellular populations.
Specifically we describe how to
i) integrate multi-modal observations, in which $\mu$ and $\nu$ represent the state of population different observable domains
and ii) model perturbation responses of cellular populations, in which $\mu$ and $\nu$ represent population state at two different time points.
A major theme throughout this thesis is the demonstration of clinical utility of the developed methods.
In particular, we apply a method developed to recover and predict heterogeneous single-cell responses to a cohort of melanoma patients and demonstrate how it can improve the analysis, understanding, and ultimately help tailor their treatments.




\newpage
\begin{otherlanguage}{ngerman}
\pdfbookmark[1]{Zusammenfassung}{Zusammenfassung}
\chapter*{Zusammenfassung}
% TODO: (later) translate abstract
[german translation]
\end{otherlanguage}


%\newpage
%\chapter*{Acknowledgments}
% TODO: (later) write awks
\endgroup

\vfill
