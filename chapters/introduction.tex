\chapter{Introduction}
\subsection{Scope}
\subsection{Publications}
\subsection{Collaborators}

\section{Individual cells as complex machinery}
\subsection{central dogma}
\subsection{pathways \& signaling hierarchies}

\section{Principles of cancer}
Cancer is a family of genetic diseases in which normal cells undergo genetic mutations that cause them to undergo unchecked growth-promoting, resulting in a mass of abnormal cells known as a tumor.
This growth can lead to serious health complications as cells
can go on to invade nearby tissues and spread to other parts of the body.
The development of effective cancer treatment strategies has been at the forefront of biological research, including moonshot projects such as \cite{need}. % eg moon shots
Many of the methods and applications explored within this thesis are motivated by problems and challenges arising from udnerstanding and treating cancer.

\subsection{hallmarks}
A major complication towards the treatment of cancer lie in the myriad mechanisms cells use to escape the natural regulatory processes that prevent unchecked growth \cite{need}.
The \emph{Hallmarks of Cancer}, proposed by \citeauthor{foo} in \citeyear{foo} and updated in \citeyear{foo} and \citeyear{foo} is a framework designed to understand the key capabilities and enabling characteristics that are exhibited by cancer cells.
The framework describes eight hallmark capabilities,
"Sustaining proliferative signaling", 
"Evading growth suppressors",
"Avoiding immune destruction",
"Enabling replicative immortality",
"Activating invasion \& metastasis",
"Inducing angiogensis",
"Resisting cell death",
and "Deregulatin cellular energetics."
These hallmark capabilities can be considered as "gates" in the sense that they each represent some core regulator process to prevent cancer cell activity.
The framework also introduces two enabling characteristics, "Genome instability \& mutation" and "Tumor-promoting inflammation", that describe the class of mechanisms by which a cell can pass each "gate" and obtain the function of a hallmark characteristic.
In essensce, the framework poses that cancer cells, by way of genetic mutations or tumor-promoting inflammation, both
i) sustain, promote, or generate cell growth signals, and, simultaneously,
ii) ignore or resist cell death and cell growth supression signals.

Below some example mechanisms underlying a subset of the hallmarks are described.

\paragraph{Sustaining proliferative signaling}
\begin{enumerate}
  \item normal cells control growth-promoting signals
  \item cancer cells deregulate these signals
  \item keep growth signals active
  \item signals are in the form of "growth factors that bind cell-surface receptors," e.g. "tyrosine kinase domains"
  \item (somatic) mutations: BRAF \cite{davies2010}, PI3K \cite{jiang2009,yuan2008}
  \item mTOR % TODO: read more about this
\end{enumerate}


\paragraph{Evading growth suppressors}
\begin{enumerate}
  \item learn to evade death signals
\end{enumerate}

\paragraph{Resisting cell death}

\paragraph{Enable replicative immortality}

\paragraph{Induce angiogensis}

\paragraph{Activate invasion and metastasis}

\subsection{TME}


\section{Single-cell profiling technologies}

\subsection{scRNA-seq}
\subsubsection{autoencoders}

\subsection{proteomic technologies}
\subsubsection{CyTOF}
\subsubsection{4i \& other imaging}
\subsubsection{marker choices}
\subsection{mention spatial}

\section{Casting as alignment}
\subsection{Holistic view of cells}
\subsection{Dynamic processes}
\subsection{Tumor profiler project}
\subsection{The observation problem}
