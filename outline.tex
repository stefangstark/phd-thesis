\chapter{Introduction}
\subsection{Scope}
\subsection{Publications}
\subsection{Collaborators}

\section{Individual cells as complex machinery}
\subsection{central dogma}
\subsection{pathways \& signaling hierarchies}
\section{ Single-cell profiling technologies}
\subsection{scRNA-seq}
\subsection{CyTOF}
\subsection{4i \& Proteomic imaging}

\section{Constructing single-cell representations}
\subsection{eg. autoencoders}

\section{Casting as alignment}
\subsection{Holistic view of cells}
\subsection{Dynamic processes}
\subsection{The observation problem}

\chapter{Multi-modal integration}

\chapter{Learning perturbation responses}
\section{Related works}
\subsection{Piece-wise linear approximations}
\subsection{Linear shifts in latent space}

\section{Optimal transport}
\subsection{Primal OT formulation}
\subsection{Dual OT formulation}
\subsection{Neural optimal transport}
\subsubsection{ICNNs}

\section{Predicting perturbation responses with CellOT}
\subsection{IID: learning cancer treatment outcomes in two modalities}
\subsubsection{utilized metrics}
\subsubsection{lack of ground truth \& qualitative analyses}

\subsection{OOD: generalizing responses}
\subsubsection{Cross species}
\subsubsection{Lupus patients}
\subsubsection{Stem cell development}

\chapter{Clinical applications}
\section{Cohort description}
\subsection{challenges, standardization,  normalization}
\subsection{heterogeneity, overview, etc}

\section{Learning patient responses}
\subsection{quantitative metrics}
\subsection{clinical associations}

\section{ Predicting patient responses}
\subsection{ quantitative metrics}
\subsection{ prediction task}

\chapter{Towards single-cell foundation models: interpretable latent spaces}
